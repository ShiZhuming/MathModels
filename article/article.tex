\documentclass[12pt,AutoFakeBold]{article}%第二个参数解决宋体没有粗体的问题
\usepackage{booktabs}%绘制三线表
\usepackage{fontspec}%控制字体
\usepackage[version=3]{mhchem}%化学反应式
\usepackage{xeCJK}%控制中文字体
\usepackage{latexsym}%绘制特殊数学符号
\usepackage[a4paper]{geometry}%纸张大小和页边距
\usepackage{pdfpages}%插入pdf页面
\usepackage{titlesec}%控制标题格
\usepackage{graphicx, epstopdf}%此处使.eps文件可转化成.pdf并应用自动编号功能
\usepackage[journal=angew]{chemstyle}%此处设置图式的样式
\usepackage{siunitx}%数学模式中使用SI单位
\usepackage{color}%提供颜色支持
\usepackage{listings,xcolor}

\geometry{left=2.5cm,right=2.5cm,top=2.5cm,bottom=2.5cm}%页边距设置
\setmainfont{Times New Roman}%英文字体
\setCJKmainfont{SimSun}%中文字体
\newfontfamily\arial{Arial}%Arial字体
\XeTeXlinebreaklocale "zh"%中文自动换行
\XeTeXlinebreakskip = 0pt plus 1pt%中文自动换行

\begin{document}
    % cover

    % main body
    
    % appendix
    由物料守恒可见
    \[[\ce{H+}]+[\ce{NH4+}]=[\ce{C2H5O-}]\]
    代入分布系数,全部化成$[\ce{H+}]$的函数,得到
    \[[\ce{H+}]+\frac{[\ce{H+}]}{[\ce{H+}]+K_a(\ce{NH4+})}\times c_0(\ce{NH4+})=\frac{K(\ce{C2H5OH})}{[\ce{H+}]}\]
    输入计算器计算即可,我身边没有计算器算不出具体的值,抱歉。
        
\end{document}